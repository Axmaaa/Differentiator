\documentclass[12pt]{article}
\usepackage{ucs}
\usepackage[utf8x]{inputenc}
\usepackage[english,russian]{babel}
\pagestyle{empty}
\usepackage{amsmath}

\begin{document}
  $$\frac{2}{4 + 2} \cdot 5^{x}$$
После упрощения получаем
    $$\frac{1}{3} \cdot 5^{x}$$
По правилу производной произведения $\left(f * g\right)' = f' * g + f * g'$, где $f = \frac{2}{4 + 2}$, а $g = 5^{x}$\\
Продифференцируем $f = \frac{2}{4 + 2}$:\\По правилу производной частного $\left(\frac{f}{g}\right)' = \frac{f' * g - f * g'}{g ^ {2}}$, где $f = 2$, а $g = 4 + 2$\\
Продифференцируем $f = 2$:\\Производная числа всегда равна нулю\\$\left(2\right)' = 0$\\
Теперь продифференцируем $g = 4 + 2$:\\По правилу производной суммы $\left(f + g\right)' = f' + g'$, где $f = 4$, а $g = 2$\\
Продифференцируем $f = 4$:\\Производная числа всегда равна нулю\\$\left(4\right)' = 0$\\
Теперь продифференцируем $g = 2$:\\Производная числа всегда равна нулю\\$\left(2\right)' = 0$\\$\left(4 + 2\right)' = 0 + 0$\\$\left(\frac{2}{4 + 2}\right)' = \frac{0 \cdot \left(4 + 2\right) - 2 \cdot \left(0 + 0\right)}{\left(4 + 2\right)^{2}}$\\
Теперь продифференцируем $g = 5^{x}$:\\По правилу производной экспоненты $\left(\alpha ^ {f}\right)' = \alpha ^ {f} * ln\left(\alpha\right) * f'$, где $\alpha = 5$, а $f = x$\\
Продифференцируем $f = x$:\\$x$ - главная переменная\\$\left(x\right)' = 1$\\$\left(5^{x}\right)' = 5^{x} \cdot ln\left(5\right) \cdot 1$\\Итак, производная равна
    $$\frac{0 \cdot \left(4 + 2\right) - 2 \cdot \left(0 + 0\right)}{\left(4 + 2\right)^{2}} \cdot 5^{x} + \frac{2}{4 + 2} \cdot 5^{x} \cdot ln\left(5\right) \cdot 1$$
После упрощения получаем
    $$\frac{1}{3} \cdot 5^{x} \cdot ln\left(5\right)$$
\end{document}